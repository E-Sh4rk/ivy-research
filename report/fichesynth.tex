
\documentclass{article}

\usepackage[francais]{babel}
\usepackage[T1]{fontenc}
\usepackage[utf8]{inputenc}
\usepackage{a4wide}
\usepackage{palatino}

\let\bfseriesbis=\bfseries \def\bfseries{\sffamily\bfseriesbis}


\newenvironment{point}[1]%
{\subsection*{#1}}%
{}

\setlength{\parskip}{0.3\baselineskip}

\begin{document}
\pagenumbering{gobble}

\title{Generalization of counterexamples for inductive invariant synthesis}

\author{Mickaël Laurent, supervised by Bryan Parno\\Carnegie Mellon University}

\date{March-July 2018}

\maketitle

\pagestyle{empty} %
\thispagestyle{empty}

%% Attention: pas plus d'un recto-verso!


\begin{point}{General context}
  
  De quoi s'agit-il? D'où vient la question? Quels sont les travaux
  déjà accomplis dans ce domaine dans le monde?

\end{point}

\begin{point}{The problem}
  
  Quelle est la question que vous avez résolue? Pourquoi est-elle
  importante, à quoi cela sert-il d'y répondre?  Pourquoi êtes-vous
  le premier chercheur de l'univers à l'avoir posée?

\end{point}

\begin{point}{Contributions}

  Qu'avez vous proposé comme solution à cette question? Attention, pas
  de technique, seulement les grandes idées! Soignez particulièrement
  la description de la démarche \emph{scientifique}.
 
\end{point}

\begin{point}{Results and guarantees}

  Qu'est-ce qui montre que cette solution est une bonne solution? Des
  expériences, des corollaires? Commentez la \emph{stabilité} de votre
  proposition: comment la validité de la solution dépend-elle des
  hypothèses de travail?

\end{point}


\begin{point}{Review and prospects}
  
  Et après? En quoi votre approche est-elle générale? Qu'est-ce que
  votre contribution a apporté au domaine? Que faudrait-il faire
  maintenant? Quelle est la bonne \emph{prochaine} question?

\end{point}


\end{document}







