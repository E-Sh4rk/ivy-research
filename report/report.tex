\documentclass[11pt,a4paper,oldfontcommands]{memoir}
\usepackage[utf8]{inputenc}
\usepackage[T1]{fontenc}
\usepackage{microtype}
\usepackage[dvips]{graphicx}
\usepackage{xcolor}
\usepackage{times}

\usepackage[
breaklinks=true,colorlinks=true,
%linkcolor=blue,urlcolor=blue,citecolor=blue,% PDF VIEW
linkcolor=black,urlcolor=black,citecolor=black,% PRINT
bookmarks=true,bookmarksopenlevel=2]{hyperref}

\usepackage{geometry}
% PDF VIEW
% \geometry{total={210mm,297mm},
% left=25mm,right=25mm,%
% bindingoffset=0mm, top=25mm,bottom=25mm}
% PRINT
\geometry{total={210mm,297mm},
left=20mm,right=20mm,
bindingoffset=10mm, top=25mm,bottom=25mm}

\OnehalfSpacing
%\linespread{1.3}

%%% CHAPTER'S STYLE
\chapterstyle{bianchi}
%\chapterstyle{ger}
%\chapterstyle{madsen}
%\chapterstyle{ell}
%%% STYLE OF SECTIONS, SUBSECTIONS, AND SUBSUBSECTIONS
\setsecheadstyle{\Large\bfseries\sffamily\raggedright}
\setsubsecheadstyle{\large\bfseries\sffamily\raggedright}
\setsubsubsecheadstyle{\bfseries\sffamily\raggedright}


%%% STYLE OF PAGES NUMBERING
%\pagestyle{companion}\nouppercaseheads 
%\pagestyle{headings}
%\pagestyle{Ruled}
\pagestyle{plain}
\makepagestyle{plain}
\makeevenfoot{plain}{\thepage}{}{}
\makeoddfoot{plain}{}{}{\thepage}
\makeevenhead{plain}{}{}{}
\makeoddhead{plain}{}{}{}


\maxsecnumdepth{subsection} % chapters, sections, and subsections are numbered
\maxtocdepth{subsection} % chapters, sections, and subsections are in the Table of Contents


%%%---%%%---%%%---%%%---%%%---%%%---%%%---%%%---%%%---%%%---%%%---%%%---%%%

\begin{document}

%%%---%%%---%%%---%%%---%%%---%%%---%%%---%%%---%%%---%%%---%%%---%%%---%%%
%   TITLEPAGE
%
%   due to variety of titlepage schemes it is probably better to make titlepage manually
%
%%%---%%%---%%%---%%%---%%%---%%%---%%%---%%%---%%%---%%%---%%%---%%%---%%%
\thispagestyle{empty}

{%%%
\sffamily
\centering
\Large

~\vspace{\fill}

{\huge 
Invariant Synthesis by Counterexample Generalization
}

\vspace{2.5cm}

{\LARGE
Mickaël LAURENT
}

\vspace{3.5cm}

Carnegie Mellon University\\
Ecole Normale Supérieure Paris-Saclay

\vspace{3.5cm}

Supervisor: Bryan PARNO

\vspace{\fill}

March-July 2018

%%%
}%%%

\clearpage%\cleardoublepage
%%%---%%%---%%%---%%%---%%%---%%%---%%%---%%%---%%%---%%%---%%%---%%%---%%%
%%%---%%%---%%%---%%%---%%%---%%%---%%%---%%%---%%%---%%%---%%%---%%%---%%%

\tableofcontents*

\clearpage

%%%---%%%---%%%---%%%---%%%---%%%---%%%---%%%---%%%---%%%---%%%---%%%---%%%
%%%---%%%---%%%---%%%---%%%---%%%---%%%---%%%---%%%---%%%---%%%---%%%---%%%

\chapter{Presentation of IVy}

    \section{Context}

    As distributed programs become more and more widespread, their verification is a major challenge.

    The general purpose of my internship is to make the certification of those programs easier.
    Certifying a program consists in writing some specifications (in our case, some safety properties), and then proving than the program satisfy them.
    Several languages and tools already exist for that:

    \begin{itemize}
        \item Proof-assistants (Coq, Agda, etc.):
        The user has to manually write the proof (altough he can benefit from some help).
        It is a long and complicated process, but once the proof is complete, the proof assistant will always accept it
        if it is correct.
        \item SMT-solver based tools (dafny, F*, etc.):
        Specifications are automatically checked using a SMT-solver. This process is undecidable in the general case,
        so there are 3 possible outputs:
        \begin{itemize}
            \item Yes, the program matches the specifications
            \item No, the program doesn't match the specifications (sometimes a counterexample can be provided)
            \item I don't know whether the program matches the specifications or not.
        \end{itemize}
        When this last case occurs, the user has to reformulate the specifications differently or to explicitely write some intermediate properties (like inductive invariants)
        in order to help the SMT-solver.
    \end{itemize}

    IVy is a language that allows the user to certify its program using a SMT-solver based approach.
    However, unlike most of its concurrents, IVy restricts the language and the logic used for specifications
    in order to be able to check the program in a decidable way.

    This approach has many advantages:
    \begin{itemize}
        \item Once the program is written and specified in an accepted logic, it can be checked more easily:
        the checker can always decide whether the program is correct or not, and give a counterexample if it is not.
        \item Our code, specifications and proof indications are not dependent on some heuristics/specificities of the SMT-solver used,
        and so we don't need any specific knowledge.
        \item If a program has been checked using a certain version IVy, it can also be checked with
        any future version (because it does not depend on any heuristic).
    \end{itemize}
    
    The main disadvantage is that the user is forced to specify its program in a decidable fragment of first order logic:
    it can force him to rethink the architecture of its code, to fragment its code by creating intermediate abstract modules,
    to add some `ghost' variables, etc.

    \section{The language}

    IVy is inspired by a restricted modeling language called RML (Relational Modeling Language).

    A RML program is composed of:
    \begin{itemize}
        \item A list of uninterpreted types that we call `sorts'. A sort has a finite number of elements (`values'), but this number is unbounded.
        \item A list of variables, relations and functions over these types (a variable can be considered like a function of arity 0, and a relation can be considered as a function
        which returns a boolean). These elements are mutable and their valuation constitute the `state' of the program.
        \item A list of axioms over these variables, functions and relations (we use a fragment of the first order logic that we will detail later).
        \item A list of transitions (see the figure below).
        \item A special `init' transition can be used to constraint the initial states of the system.
    \end{itemize}

    [Insert figure EPR cmd]

    One important thing to note is that RML only allows finite uninterpreted sorts and booleans (and possibly other enumerated types).
    In particular, we can't define arithmetic operations.
    However, we will see later that it is possible to extend it with linear integer arithmetic or some other theories.

    Also, unlike many other languages like F*, IVy does not allow refinement types nor dependent types (for decidability reasons).
    As a consequence, we can't encode any property in the types.
    Instead, the specification of the program will be expressed using assertions and invariants.

    \section{Decidable logics}

        \subsection{EPR}

        \subsection{FAU}
    
    \section{Deciding IVy}

    \section{Interactive tools}

    \section{Challenges}

%\chapter{Research directions}
%    \section{Make code EPR}
%        \subsection{Fragmentation}
%        \subsection{Adding relations}
%    \section{Adding typing}
%    \section{Invariant Synthesis}

\chapter{Invariant Synthesis}

    \section{Using weakest precondition}

    \section{Counterexample generalization}

    \section{Limits}

\chapter{My contributions}

    \section{Filtering constraints using code analysis}

    \section{Model-dependent constraints}

    \section{Weakening the conjecture}

    \section{Guarantees}

    \section{Comparison to other methods}

\chapter{Conclusion}

    \section{What remains to be done}

    \section{Lessons from this internship}

\bibliographystyle{unsrt}
\bibliography{sample}

\end{document}

